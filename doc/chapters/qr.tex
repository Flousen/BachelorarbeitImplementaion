\chapter{QR factorisation}

\section{LAPACK}
Mathe \cite{DGEQR2}
\begin{align}
	H &= I - \tau \omega \omega^T \\
	\tau &= \frac{\alpha - \beta}{\beta} \\
	\alpha &= A(i,i)\\
	\beta &= \text{sign}(\alpha) \left|\sqrt{\alpha^2 + \|x\|^2}\right|\\
	x &= A(i+1:m,i)\\
	\omega &= A(i+1:m,i) * \frac{1}{\alpha - \beta}
\end{align}
Algorithmus
\begin{lstlisting}
householderVektor(Vektor v, alpha, tau)
  beta = sign(sqrt(alpha ^2 + norm(x)^2),alpha)
  tau = (alpha - beta) / beta	
  scal(1/(alpha - beta), v)
\end{lstlisting}
\begin{lstlisting}
tau=zeros(min(m,n))
for i = 0 : min(m,n)
  householderVektor(A(i+1:m,i), A(i,i), tau(i)) 
  if (i < n && tau != 0)
    AII = A(i,i)
    A(i,i)= 1
    A = A - tau *w(w'*A) // MV und rank1
    A(i,i) = AII
\end{lstlisting}

\section{NUM3 Urban}
Mathe
\begin{align}
	H &= I - 2 \frac{\omega \omega^T}{\omega^T \omega}\\
	\omega_1 &= \frac{x - \alpha e_1}{x_1 - \alpha}\\
	\alpha ^2 &= \|x\|^2 
\end{align}
Algorithmus
\begin{lstlisting}
householderVektor(Vektor x, omega, beta)
  n = length(x)
  if n> 1
    sigma = x(2:end)'*x(2:end);
    if sigma==0
      beta = 0;
    else
      mu = sqrt(x(1)^2+sigma);
      if x(1)<=0
        tmp = x(1) - mu;
      else
        tmp = -sigma / (x(1) + mu);
      end
      beta = 2*tmp^2/(sigma + tmp^2);
      x(2:end) = x(2:end)/tmp;
    end
    v = [1;x(2:end)];
  else
    beta = 0;
    v = 1;
  end
\end{lstlisting}

\begin{lstlisting}
for i = i:n
  housevector(A(i:m, i), w, beta)
  A(i:m,i:n) = (I(m-i+1) - beta * w * w')*A(i m,i:n)
  if i < m
    A(i + 1 : m, i) = w(2:m-i+1)
\end{lstlisting}



\section{Fragen}
\begin{enumerate}
	\item Was mach ich bei LAPACK mit dem $ \tau $ ?
\end{enumerate}
